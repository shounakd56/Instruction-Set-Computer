\documentclass[10pt]{article}

\usepackage[a4paper, left=2cm, right=2cm]{geometry} % A4 paper size and thin margins
\usepackage{graphicx}
\usepackage{float}
\usepackage{hyperref}
\usepackage{float}

\usepackage[utf8]{inputenc}
\usepackage[T1]{fontenc}
\usepackage{parskip}
\usepackage{media9}

\usepackage{fancyhdr}
\pagestyle{fancy}
\fancyhf{}
\lhead{Experiment 10} 
\rhead{Section \thesection}
\rfoot{Page \thepage}

\usepackage{xcolor} % Required for specifying custom colours
\definecolor{grey}{rgb}{0.9,0.9,0.9} % Colour of the box surrounding the title

\usepackage[utf8]{inputenc} % Required for inputting international characters
\usepackage[T1]{fontenc} % Output font encoding for international characters
%\usepackage[sfdefault]{ClearSans} % Use the Clear Sans font (sans serif)
%\usepackage{XCharter} % Use the XCharter font (serif)

\usepackage{xcolor}
\hypersetup{
    colorlinks,
    linkcolor={red!50!black},
    citecolor={blue!50!black},
    urlcolor={blue!80!black}
}

\begin{document}

\begin{titlepage} % Suppresses displaying the page number on the title page and the subsequent page counts as page 1
	
	%------------------------------------------------
	%	Grey title box
	%------------------------------------------------
	
	\colorbox{grey}{
		\parbox[t]{0.93\textwidth}{ % Outer full width box
			\parbox[t]{0.91\textwidth}{ % Inner box for inner right text margin
				\raggedleft % Right align the text
				\fontsize{50pt}{80pt}\selectfont % Title font size, the first argument is the font size and the second is the line spacing, adjust depending on title length
				\vspace{0.7cm} % Space between the start of the title and the top of the grey box
				
				\textbf{EE-224 Project: CPU}  \hspace{4pt}
				
				\vspace{0.7cm} % Space between the end of the title and the bottom of the grey box
			}
		}
	}
	
	\vfill % Space between the title box and author information
	
	%------------------------------------------------
	%	Author name and information
	%------------------------------------------------
	
	\parbox[t]{0.93\textwidth}{ % Box to inset this section slightly
		\raggedleft % Right align the text
		\large % Increase the font size
        \hfill\rule{0.2\linewidth}{1pt} \\
		{\Large Vinay Sutar(21d070078) \\ Shounak Das(21d070068) \\ Aditya Anand(21d070007) \\ Parth Arora(21d070047) \\ Daksh Pakal(210070023)}\\[4pt] % Extra space after name
        \hfill\rule{0.2\linewidth}{1pt} \\
		  November 2022\\[4.8pt] % Extra space before URL
		
		\hfill\rule{0.2\linewidth}{1pt}% Horizontal line, first argument width, second thickness
	}
	
\end{titlepage}
\newpage
\restoregeometry 

\tableofcontents

\newpage
\Large
\section{\textbf{Instructions}}

\subsection{\textbf{\color{blue}{Types}}}
\begin{figure}[H]
    \centering
    \includegraphics{Screenshot 2022-11-30 012028.png}
    \caption{Types of Instructions}
    \label{fig:my_label1}
\end{figure}

\subsection{\textbf{\color{blue}{Instructions}}}
\begin{figure}[H]
    \centering
    \includegraphics{Screenshot 2022-11-30 012059.png}
    \caption{Instructions}
    \label{fig:my_label2}
\end{figure}

\subsection{\textbf{\color{blue}{Instruction Description}}}
\begin{figure}[H]
    \centering
    \includegraphics{Screenshot 2022-11-30 144523.png}
    \label{fig:my_label6}
\end{figure}

\begin{figure}[H]
    \centering
    \includegraphics{Screenshot 2022-11-30 144543.png}
	\caption{Instruction Description}
    \label{fig:my_label7}
\end{figure}

\newpage

\section{\textbf{Components}}

\subsection{\textbf{\color{blue}{Instruction Register}}}
Memory from the register is read (\textit{irw='1'}) and stored as a signal \textit{instruction}. The first 4 bits (15 downto 12) of the instruction is the \textit{opcode}, the next 3 bits is the \textit{register A} (11 downto 9), the next 3 bits is the \textit{register B} (8 downto 6), the next 3 bits is the \textit{register C} (5 downto 3), 3rd bit is unused and (1 downto 0) i.e. the last 2 bits correspond to the \textit{carry and zero flag}. 

\subsection{\textbf{\color{blue}{Memory/RAM}}}
The input variable \textit{init} when goes high, leads to the instruction (15 downto 0) (16 bits) to be read into storage element named $ store_ram $
The instruction contains information about opcode, registers and carry and zero flags.
\\ The variable \textit{mread} going high(='1') results in reading of data($ store_ram $) as \textbf{dout}, while \textit{mwrite} corresponds to writing of data in the variable \textbf{din} for further processing.

\subsection{\textbf{\color{blue}{Register File}}}
The \textit{registerwrite} variable allows the data to be transfered from the input $ din_m $ to the register. Data from 2 registers is required to execute the desired instruction. These 2 registers viz \textit{Register A} and \textit{Register B} (out of 8) are chosen depending on the values of variable \textit{registerselector}. The data from the instruction is sent to the selected registers for further processing in the ALU.

\subsection{\textbf{\color{blue}{ALU}}}
Here various processes (like add, subtract, multiply etc) are performed on the 2 inputs (\textit{inp1, inp2}).
Selector is a variable that triggers the operation to be performed on the inputs.
The output is then used to decide carry and zero flag as shown in the code.

\subsection{\textbf{\color{blue}{RISC}}}
This is the main file connecting all the components.
Initially the states are initialized in order as constant signals. All the individual components are included here. Further the opcodes of the instructions are initialized as written in the code.

\subsection{\textbf{\color{blue}{Sequencer}}}

\subsection{\textbf{\color{blue}{Sign Extender}}}

\subsection{\textbf{\color{blue}{Sixteen Bit Adder}}}

\newpage

\section{\textbf{Dataflow}}

\begin{figure}[H]
    \centering
    \includegraphics[width=18cm,height=14cm]{Screenshot 2022-11-30 013403.png}
	\caption{Dataflow}
    \label{fig:my_label3}
\end{figure}

\newpage

\section{\textbf{Implementation}}
The memory component has the following ‘in’ ports: 
\begin{enumerate}
\item state: denoting the state of the machine 
\item init: to store predefined instructions in the memory 
\item mr: bit for memory read 
\item mw: bit for memory write 
\item dataPointer: denoting address in the memory 
\item di: data in
It has only one ‘out’ port:
\begin{enumerate}
\item do: data out
\end{enumerate}
\end{enumerate}

\section{\textbf{RTL Diagram}}

\begin{figure}[H]
    \centering
    \includegraphics[width=18cm,height=18cm]{Screenshot 2022-11-30 173041.png}
	\caption{RTL Diagram}
    \label{fig:my_label4}
\end{figure}

\section{\textbf{Technology Viewer Diagram}}

\begin{figure}[H]
    \centering
    \includegraphics[width=18cm,height=10cm]{Screenshot 2022-11-30 172949.png}
    \caption{Technology Viewer Diagram}
    \label{fig:my_label8}
\end{figure}

\section{Waveform}

\begin{figure}[H]
    \centering
    \includegraphics[width=19cm,height=6cm]{Screenshot 2022-11-30 143239.png}
	\caption{Waveform}
    \label{fig:my_label5}
\end{figure}

\end{document}



